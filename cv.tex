%%%%%%%%%%%%%%%%%%%%%%%%%%%%%%%%%%%%%%%%%
% Freeman Curriculum Vitae
% XeLaTeX Template
% Version 3.0 (September 3, 2021)
%
% This template originates from:
% https://www.LaTeXTemplates.com
%
% Authors:
% Vel (vel@LaTeXTemplates.com)
% Alessandro Plasmati
%
% License:
% CC BY-NC-SA 4.0 (https://creativecommons.org/licenses/by-nc-sa/4.0/)
%
%!TEX program = xelatex
% NOTE: this template must be compiled with XeLaTeX rather than PDFLaTeX
% due to the custom fonts used. The line above should ensure this happens
% automatically, but if it doesn't, your LaTeX editor should have a simple toggle
% to switch to using XeLaTeX.
% 
%%%%%%%%%%%%%%%%%%%%%%%%%%%%%%%%%%%%%%%%%

%----------------------------------------------------------------------------------------
%	PACKAGES AND OTHER DOCUMENT CONFIGURATIONS
%----------------------------------------------------------------------------------------

\documentclass[
	9pt, % Default font size, can be between 8pt and 12pt
]{FreemanCV}

\columnratio{0.50, 0.5} % Widths of the two columns, specified here as a ratio summing to 1 to correspond to percentages; adjust as needed for your content 

% Headers and footers can be added with the following commands: \lhead{}, \rhead{}, \lfoot{} and \rfoot{}
% Example right footer:
%\rfoot{\textcolor{headings}{\sffamily Last update: \today. Typeset with Xe\LaTeX}}

%----------------------------------------------------------------------------------------

\begin{document}

\begin{paracol}{2} % Begin two-column mode

%----------------------------------------------------------------------------------------
%	YOUR NAME AND CURRICULUM VITAE TITLE
%----------------------------------------------------------------------------------------

\parbox[][0.08\textheight][c]{\linewidth}{ % Box to hold your name and CV title; change the fixed height as needed to match the colored box to the right
	\centering % Horizontally center text
	
	\vspace*{\fill}
	\sffamily\Huge Artemis Livingstone  % Your name
	\vspace*{\fill}
	\vfill
	
}

%----------------------------------------------------------------------------------------
%	WORK EXPERIENCE
%----------------------------------------------------------------------------------------

\section{Work Experience}

% Each job is added with a \jobentry command. Below is an empty one to use as a template:

%\jobentry
%	{} % Duration
%	{} % FT/PT (full time or part time)
%	{} % Employer
%	{} % Job title
%	{} % Description

% All 5 parameters must be supplied but any can be empty if you don't need them

%------------------------------------------------

\jobentry
	{Current, from Sept 2022} % Duration
	{FT} % FT/PT (full time or part time)
	{L3Harris} % Employer
	{Software Engineer and Security Intern} % Job title
	{Details of role and projects are under NDA.} % Description

%------------------------------------------------

\jobentry
	{June 2022 -- Sept 2022} % Duration
	{FT} % FT/PT (full time or part time)
	{D. E. Shaw \& Co.} % Employer
	{Systems Intern} % Job title
	{\begin{itemize}[noitemsep, topsep=0pt, partopsep=0pt, leftmargin=8pt, before =\leavevmode\vspace*{-\baselineskip}]
			\item Developed policy for credential distribution and self-service credential resetting tools followed by an implementation in Rust served using k8s deploying via helmcharts.
			\item Developed libraries to allow for Kerberos authentication as middleware and guards in the Rust-based Rocket.rs and Actix frameworks with a fellow intern.
			\item  Added Kerberos support to internal Rust libraries, enabling communication with internal HashiCorp Vault instance. 
			\item Identified time sinks in common VOD tasks then implemented 
			PowerShell and Bash tooling for them by reverse engineering the admin API for Cisco Webex. 
			This included a prototype for bulk recording search and download as well as scriptable user permission updates. 
	\end{itemize}} % Description
%------------------------------------------------

\jobentry
	{June 2021 -- June 2022} % Duration
	{FT/PT} % FT/PT (full time or part time)
	{Sunday Group Inc} % Employer
	{Security Consultant and Developer} % Job title
	{
	Sunday Group are a blockchain development group working on developing a trust-based authentication 
	system for proof of work based crypto currency. 
	\medspace

	\begin{itemize}[noitemsep, topsep=0pt, partopsep=0pt, leftmargin=8pt]
		\item	Developed Merkle Tree based authentication system in Rust.
		\item	Audited memory unsafe code for back-end server system.
		\item	Developed set of fuzzing tools to help target a JSON RPC server.
		\item	Developed and tested the trust-based authentication system from white papers in Rust.
		\item	Developed and deployed containerized instances to allow for software to run regardless of hardware. 
	\end{itemize}
	} % Description

%----------------------------------------------------------------------------------------
%	Projects
%----------------------------------------------------------------------------------------

\section{Projects}
\devproject{Functionally Programmed Discord Bot}{Haskell}
{
	\begin{itemize}[noitemsep, topsep=0pt, partopsep=0pt, leftmargin=8pt, before =\leavevmode\vspace*{-\baselineskip}]
		\item Created a Discord bot that serves over 500 users with various useful learning resources.
		\item Used Docker containers to administer deployment on several servers.
		\item Researched and implemented monadic interactions for file IO and reacting to events with persistent data.
	\end{itemize}
}

\devproject{CTF Focussed DNS Server}{Rust}{
		\begin{itemize}[noitemsep, topsep=0pt, partopsep=0pt, leftmargin=8pt, before =\leavevmode\vspace*{-\baselineskip}]
			\item Created a concurrent DNS server using Rust.
			\item Used asynchronous code as well as threading to deal with requests concurrently. 
			\item Included Exfiltration and Infiltration capabilities, as well as facilitating for DNS rebinding attacks by utilizing very low caching times.
		\end{itemize}
}

\section{Awards}

% This section is laid out using a table. A \tableentry command adds lines with the following parameters:

%\tableentry{Heading}{Content}{spaceafter}
% All 3 parameters must be supplied but any can be empty if you don't need them
% A "spaceafter" value in the third parameter will add some vertical space -- this is to be used between headings, leave it empty for no extra space

%------------------------------------------------

\begin{supertabular}{r l} % Start a table with two columns, the table will ensure everything is aligned
		
	\tableentry{2023}{\textbf{RET2 Wargames Training Completed}}{}
	\tableentry{}{\textit{RET2}}{spaceafter}

 
	\tableentry{2022}{\textbf{6\textsuperscript{th} Place at DEFCON 30 CTF }}{}
	\tableentry{}{\textit{Organizers}}{spaceafter}
 
	\tableentry{2021}{\textbf{SANS Foundations Certification}}{}
	\tableentry{}{\textit{GIAC}}{spaceafter}

	\tableentry{2021}{\textbf{1\textsuperscript{st} Place in Hack the Box University CTF}}{}
	\tableentry{}{\textit{SIGINT}}{spaceafter}
	

	
\end{supertabular}

% %----------------------------------------------------------------------------------------
% %	REFERENCES
% %----------------------------------------------------------------------------------------

% \section{References}

% %\textit{References available on request} % Uncomment if you'd rather not include references and remove the section below

% %------------------------------------------------

% % This section is laid out using a table. A \tableentry command adds lines with the following parameters:

% %\tableentry{Heading}{Content}{spaceafter}
% % All 3 parameters must be supplied but any can be empty if you don't need them
% % A "spaceafter" value in the third parameter will add some vertical space -- this is to be used between headings, leave it empty for no extra space

% %------------------------------------------------

% \begin{supertabular}{r l} % Start a table with two columns, the table will ensure everything is aligned
	
% 	%------------------------------------------------
	
% 	\tableentry{}{\textbf{Dr. Isaac Kleiner}}{spaceafter}
% 	\tableentry{Position}{Professor}{}
% 	\tableentry{Employer}{\href{https://web.mit.edu/physics/}{Department of Physics}}{}
% 	\tableentry{}{\href{https://web.mit.edu}{\textit{Massachusetts Institute of Technology}}}{spaceafter}
% 	\tableentry{Phone}{+1 (617) 253 1000 x5322 (Work)}{}
% 	\tableentry{Mobile}{+1 (232) 842-3583}{}
	
% 	%------------------------------------------------
	
% 	\\ % Additional vertical whitespace between the references
	
% 	%------------------------------------------------
	
% 	\tableentry{}{\textbf{Dr. Eli Vance}}{spaceafter}
% 	\tableentry{Position}{Scientist (HL1)}{}
% 	\tableentry{Employer}{\href{http://www.bmrf.us}{Black Mesa Research Facility}}{spaceafter}
% 	\tableentry{Email}{\href{mailto:e.vance@bmrf.us}{e.vance@bmrf.us}}{}
% 	\tableentry{Phone}{+1 (800) 786-1410 x6235 (Work)}{}
% 	\tableentry{Mobile}{+1 (201) 632-3901}{}
	
% 	%------------------------------------------------
	
% \end{supertabular}

% \medskip % Extra vertical whitespace before the next section

%----------------------------------------------------------------------------------------

\switchcolumn % Switch to the second (right) column

%----------------------------------------------------------------------------------------
%	COLORED CONTACT DETAILS BOX
%----------------------------------------------------------------------------------------

\parbox[top][0.08\textheight][c]{\linewidth}{ % Box to hold the colored box; change the fixed height as needed to match the box to the left
	\colorbox{shade}{ % Create colored box and specify background color
		\begin{supertabular}{@{\hspace{3pt}} p{0.05\linewidth} | p{0.775\linewidth}} % Start a table with two columns, the table will ensure everything is aligned
			\raisebox{-1pt}{\faPhone} & +44 7902 160039 \\ % Phone number
			\raisebox{-1pt}{\small\faEnvelope} & \href{mailto:ar.d.livingstone@gmail.com}{ar.d.livingstone@gmail.com} \\ % Email address
			% \raisebox{-1pt}{\faGithub} & \href{https://github.com/social-anthrax}{https://github.com/social-anthrax} \\ % GitHub profile
			\raisebox{-1pt}{\faLinkedinSquare} & \href{https://www.linkedin.com/in/artemis-livingstone/}{https://www.linkedin.com/in/artemis-livingstone/} \\ % LinkedIn profile
			\raisebox{-1pt}{\small\faDesktop} & \href{https://blog.anthrax.social}{blog.anthrax.social} \\ % Website
			% See fontawesome.pdf in the Fonts folder for all icons you can use
		\end{supertabular}
	}
	\vfill % Push content to the top of the box
}

%----------------------------------------------------------------------------------------
%	EDUCATION
%----------------------------------------------------------------------------------------

\section{Education} 

% Each qualification entry is added with a \qualificationentry command. Below is an empty one to use as a template:

%\qualificationentry
%	{} % Duration
%	{} % Degree
%	{} % Honors, achievements or distinctions (e.g. first class honors)
%	{} % Department
%	{} % Institution

% All 5 parameters must be supplied but any can be empty if you don't need them

%------------------------------------------------

\qualificationentry
	{2020 -- 2025} % Duration
	{BSc (Hons) Computer Science} % Degree
	{Expected First Class Honours.} % Honors, achievements or distinctions (e.g. first class honors)
	{} % Department
	{The University of Edinburgh} % Institution
	{
		\begin{itemize}[noitemsep, topsep=0pt, partopsep=0pt, leftmargin=8pt, before =\leavevmode\vspace*{-\baselineskip}]
			\item Object-Oriented Programming (97\%): imperative and \\object-oriented programming in Java.
			\item Computer Systems (93\%): principles of computer hardware, processor architecture, low-level programming in C. 
			\item Introduction to Algorithms and Data Structures (81\%): theory, practice, and runtime analysis of algorithms and data structures.
		\end{itemize}
	} % Description

%------------------------------------------------

\qualificationentry
	{2023 -- 2024} % Duration
	{Incoming Exchange Student, Computer Science} % Degree
	{} % Honors, achievements or distinctions (e.g. first class honors)
	{} % Department
	{The University of Pennsylvania} % Institution
	{}

%------------------------------------------------

%----------------------------------------------------------------------------------------
%  EXTRA CURRICULAR	
%----------------------------------------------------------------------------------------
\section{ExtraCurricular}

%------------------------------------------------

\extracurricularentry
	{Sept 2020 -- Sept 2022}
	{Edinburgh Computer Society}
	{President}
	{
		Elected as First Year Representative for 20/21 and President for 21/22. 
		Led committee of 11: managed budget (>£30,000), corresponded with sponsors, organized events.
		Administrated a digital social space with 500+ members, where students can interact with one another and get help from older students and graduates. 
	}

% ------------------------------------------------

\extracurricularentry{Sept 2021 -- June 2022}{Tardis Project}{Head Systems Administrator}{
	Tardis is a student-run organization at the University of Edinburgh, providing self-hosted services for 200+ students as well as a space for gaining hands-on experience in systems administration. 
	\begin{itemize}[noitemsep, topsep=0pt, partopsep=0pt, leftmargin=8pt]
		\item Deploying Grafana monitoring via Nix using Digga. 
		\item Managing a budget of £3,000 to keep Tardis upgraded and in line with safety standards.
		\item Oversaw migration from university to private location.
		\item Currently working on setting up a Kubernetes cluster.
	\end{itemize}
}

\extracurricularentry{Sept 2020 -- Sept 2022}{SIGINT, Edinburgh's InfoSec Society}
	{Social Secretary}
	{
		Organized and created challenges for PwnEd 2, 3 and 4; some of the biggest university CTFs in the UK (250+ participants).
	}

\extracurricularentry{Current, From March 2022}{Organizers}{Player}
{
		Play CTF as part of Organizers, the top team in 2022. Played in over 10 high-ranking CTFs in the 2022 season.
}

%----------------------------------------------------------------------------------------
%	AWARDS
%----------------------------------------------------------------------------------------


%----------------------------------------------------------------------------------------
%	SKILLS
%----------------------------------------------------------------------------------------

\section{Skills} 

% This section is laid out using a table. A \tableentry command adds lines with the following parameters:

%\tableentry{Heading}{Content}{spaceafter}
% All 3 parameters must be supplied but any can be empty if you don't need them
% A "spaceafter" value in the third parameter will add some vertical space -- this is to be used between headings, leave it empty for no extra space

%------------------------------------------------

\begin{supertabular}{r l} % Start a table with two columns, the table will ensure everything is aligned
	
	% %------------------------------------------------
	
	% \tableentry{Beginner}{Java, MS DOS}{spaceafter}
	
	% %------------------------------------------------
	
	\tableentry{Prog. Languages}{Rust, Python, Haskell, Java.}{spaceafter}
	
	%------------------------------------------------
	
	\tableentry{Tools}{Linux, Bash scripting, Powershell, Docker.}{}
	\tableentry{}{Experience with CI tools and Kubernetes.}{spaceafter}
	
	%------------------------------------------------

	\tableentry{Cyber Security}{Broad experience of web and web-app}{}
	\tableentry{}{based exploits.}{}
	\tableentry{}{iOS App Teardowns and Internals.}{spaceafter}

	%------------------------------------------------

	\tableentry{Languages (Fluent)}{English, Russian.}{}

	%------------------------------------------------
	
\end{supertabular}

% %----------------------------------------------------------------------------------------
% %	PUBLICATIONS
% %----------------------------------------------------------------------------------------

% \section{Publications}

% %------------------------------------------------

% \textbf{Freeman, G. R.} (1996). Chemistry of Multiply Charged Negative Molecular Ions and Clusters in the Gas Phase:  Terrestrial and in Intense Galactic Magnetic Fields. \textit{The Journal of Physical Chemistry}, \textit{100}(11), 4331-4338.

% \medskip % Vertical whitespace

% Jacobsen, F. M., Gee, N., \textbf{Freeman, G. R.} (1986). Electron mobility in liquid krypton as function of density, temperature, and electric field strength. \textit{Physical Review A}, \textit{34}(3): 2329-2335.

% \medskip % Vertical whitespace

% %------------------------------------------------

% % As an alternative to a long-form publication list, you can create a shorter summary using only DOI values and years.

% % Example \doipublication{} command to add another publication:

% %\doipublication{Year}{DOI}{firstauthor}{spaceafter}

% % All four parameters are required (can be empty though)
% % A value of "firstauthor" in the third parameter will output the DOI in bold
% % A "spaceafter" value in the fourth parameter will add some vertical space -- this is to be used between years

% %------------------------------------------------

% \subsection{Publications by DOI}

% \begin{supertabular}{r l} % Start a table with two columns, the table will ensure everything is aligned
	
% 	%------------------------------------------------
	
% 	\doipublication{1996}{10.1021/jp951483+}{firstauthor}{spaceafter}
	
% 	%------------------------------------------------
	
% 	\doipublication{1990}{10.1139/p90-097}{firstauthor}{spaceafter}
	
% 	%------------------------------------------------
	
% 	\doipublication{1986}{10.1139/v86-297}{}{}
% 	\doipublication{}{10.1103/PhysRevA.34.2329}{}{spaceafter}
	
% 	%------------------------------------------------
	
% 	& \textit{First author publications in} \textbf{bold}\\
	
% 	%------------------------------------------------
	
% \end{supertabular}

% \medskip % Extra whitespace before the next section

%----------------------------------------------------------------------------------------

\end{paracol} % End two-column mode

%----------------------------------------------------------------------------------------

\end{document}
